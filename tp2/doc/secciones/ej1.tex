\section{Problema 1: Laberinto}
\subsection{Introducción}

El escenario del primer problema es sencillo: dada una posición de arranque en un laberinto, queremos encontrar el costo del camino mínimo hacia una posición final. Para lograr esto podemos derribar hasta cierta cantidad, ya determinada, de paredes del laberinto mismo. El input total del problema consiste de:

    \begin{itemize}
        \item Una matriz $M$ de caracteres que representan el mapa, de dimensión $F\times C$ (pasados por parámetro) donde $M_{ij} \in \{$'$\cdotp$'$,$'$\#$'$,$'$o$'$,$'$x$'$\}$ según sea, respecto del orden listado del conjunto, un lugar caminable, pared, punto de origen o punto de destino. Se asume que existe un único caracter $'o'$ y un único $'x'$ en todo el mapa.
        \item Un entero $P$ que representa la cantidad máxima de paredes que podremos romper, es decir, caracteres $'\#'$ que pueden conformar el camino mínimo cuya longitud buscamos.
    \end{itemize}

También sabemos por enunciado que los caracteres del borde del mapa corresponden a una pared. Lo que quiere decir que $i \in \{0, F-1\} \ \vee \ j \in \{0, C-1\} \Rightarrow M_{ij} =\ $'$\#$'.

Teniendo en cuenta que cada movimiento realizado debe ser vertical u horizontal (los vecinos de $M_{i,j}$ son los elementos en rango del conjunto $\{M_{i+1,j},\ M_{i-1,j},\ M_{i,j+1},\ M_{i,j-1}\}$), podemos considerar el output del problema como la mínima longitud de las secuencias válidas de vecinos en el mapa, empezando por el punto de origen (sin contarlo, dado que no es una ubicación a recorrer, sino que es la primer ubicación 'recorrida' del mapa) y terminando en el punto de destino, con cantidad de caracteres $'\#'$ menor o igual a $P$.
\\

    \begin{itemize}
        \item $ M_{ij} = $'$ o $'$ $
        \item $A = \Big\{ \langle {M_{i_0, j_0}\ ...\ M_{i_{k-1}, j_{k-1}}} \rangle \ \Big | \kern 0.2cm
        M_{i_0,j_0} \in \{M_{i+1,j}\ M_{i-1,j}\ M_{i,j+1}\ M_{i,j-1}\} \kern 0.3cm \wedge \kern 0.3cm \sum\limits_{x=0}^{k-1} \beta(M_{i_x j_x} = $'$\#$'$) \leq P \kern 0.2cm \wedge \kern 0.2cm
        M_{i_{k-1}, j_{k-1}} = $'$\#$'$ \kern 0.3cm \wedge \kern 0.3cm
        \forall\ x \in [0..k-1) \kern 0.3cm M_{i_{(x+1)},j_{(x+1)}} \in \{M_{i_x+1,j_x}\ M_{i_x-1,j_x}\ M_{i_x,j_x+1}\ M_{i_x,j_x-1}\} \bigcap M_{[1..F]\times[1..C]} \Big\}$
        \item $Output = \min\limits_{C \in A}\ |C|$
    \end{itemize}

Por ejemplo, sean los siguientes valores para $M$ y $P$:
\\
    \begin{center}
        $P=0, \kern 1cm
        M =
        \begin{bmatrix}
            \# & \# & \# & \# & \# \\
            \# & . & . & . & \# \\
            \# & \mathbf{o} & \# & \mathbf{x} & \# \\
            \# & . & . & . & \# \\
            \# & \# & \# & \# & \#
        \end{bmatrix}
        \kern 1cm
        (F = 5,\ C = 5)
        $

    \end{center}

Las secuencias de longitud mínima ($4$ en este caso) son los dos únicos caminos sin repetir ubicaciones: $\langle {M_{22}, M_{23}, M_{24}, M_{34}} \rangle$ y $\langle {M_{42}, M_{43}, M_{44}, M_{34}} \rangle$.

Para ilustrar un poco la idea del parámetro $P$ veamos también este caso:
\\
    \begin{center}
        $P=1, \kern 1cm
        M =
        \begin{bmatrix}
            \#  &   \#         &    \#    &   \#         &   \# & \#    \\
            \#  &   \mathbf{o}  &    \#     & \#          & \mathbf{x}    &   \#    \\
            \#  &   .         &    \#    &    .          &   .  & \#   \\
            \#  &   .          &    .     &   .          &   . & \#    \\
            \#  &   \#         &    \#    &   \#         &   \# & \#
        \end{bmatrix}
        \kern 1cm
        (F = 5,\ C = 6)
        $
    \end{center}

Si bien el camino más corto (obviando paredes) es $\langle {M_{23}, M_{24}, M_{25}} \rangle$, la suma de paredes atravesadas es $2 > P = 1$, por lo que tendremos que conformarnos la longitud $5$ del camino  $\langle {M_{32}, M_{33}, M_{34}, M_{35}, M_{25}} \rangle$
