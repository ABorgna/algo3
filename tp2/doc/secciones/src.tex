\section{Códigos Fuente}
\subsection{Ejercicio 1}

\lstset{language=C++,
                basicstyle=\ttfamily,
                keywordstyle=\color{blue}\ttfamily,
                stringstyle=\color{red}\ttfamily,
                commentstyle=\color{green}\ttfamily,
                morecomment=[l][\color{magenta}]{\#}
}

\begin{lstlisting}[ basicstyle=\small]

#include <dungeon.h>
#include <interfaz.h>

#include <math.h>
#include <stdint.h>
#include <algorithm>
#include <cassert>
#include <iostream>
#include <map>
#include <queue>
#include <set>
#include <tuple>
#include <vector>

using namespace std;

typedef struct point_t {
    int64_t x;
    int64_t y;

    bool operator==(const struct point_t &o) const {
        return tie(x, y) == tie(o.x, o.y);
    }
} point;

int64_t n, m, pmax;
vector<vector<bool>> mapa;  // True == pared
point origen;
point destino;

void agregarCandidato(queue<pair<point, int64_t>> &candidatos,
                      vector<vector<vector<int64_t>>> &dist, point v, int64_t p,
                      int64_t d);

int64_t bfs() {
    vector<vector<vector<int64_t>>> dist(
        m, vector<vector<int64_t>>(n, vector<int64_t>(pmax + 1, -1)));

    queue<pair<point, int64_t>> candidatos;

    // Inicializar el primer lugar
    dist[origen.y][origen.x][0] = 0;
    candidatos.emplace(origen, 0);

    while (!candidatos.empty()) {
        auto cand = candidatos.front();
        candidatos.pop();
        point u = cand.first;
        int64_t p = cand.second;
        int64_t d = dist[u.y][u.x][p];

        if (u == destino) {
            return d;
        }

        // Agregamos los vecinos, si no estan visitados
        agregarCandidato(candidatos, dist, {u.x - 1, u.y}, p, d);
        agregarCandidato(candidatos, dist, {u.x + 1, u.y}, p, d);
        agregarCandidato(candidatos, dist, {u.x, u.y - 1}, p, d);
        agregarCandidato(candidatos, dist, {u.x, u.y + 1}, p, d);
    }

    return -1;
}
\end{lstlisting}
\begin{lstlisting}[ basicstyle=\small]
void agregarCandidato(queue<pair<point, int64_t>> &candidatos,
                      vector<vector<vector<int64_t>>> &dist, point v, int64_t p,
                      int64_t d) {
    // Nunca agregamos los bordes del mapa
    if (v.x <= 0 or v.y <= 0 or v.x >= n - 1 or v.y >= m - 1)
        return;

    // Si hay una pared en v nos cuesta una demolicion mas llegar
    if (mapa[v.y][v.x]) {
        p++;
        if (p > pmax)
            return;
    }

    // Si ya está visitado no lo agregamos
    if (dist[v.y][v.x][p] != -1)
        return;

    // Lo agregamos a la cola
    candidatos.emplace(v, p);
    dist[v.y][v.x][p] = d + 1;
}
\end{lstlisting}

\newpage
\subsection{Ejercicio 2}


\begin{lstlisting}[ basicstyle=\small]
#include "include/dungeon.h"
#include "include/interfaz.h"

#include <math.h>
#include <stdint.h>
#include <algorithm>
#include <cassert>
#include <iostream>
#include <map>
#include <queue>
#include <set>
#include <tuple>
#include <vector>
#include <list>

using namespace std;

int64_t n, m;
vector<vector<int64_t>>
    mapa;  // -1 = pared dura, 0 = sala, >0 = pared destruible

typedef struct point_t {
    int64_t x;
    int64_t y;

    point_t operator=(const struct point_t &o) {
        x = o.x;
        y = o.y;
        return *this;
    }
    bool operator==(const struct point_t &o) const {
        return tie(x, y) == tie(o.x, o.y);
    }
    bool operator!=(const struct point_t &o) const {
        return tie(x, y) != tie(o.x, o.y);
    }
    bool operator<(const struct point_t &o) const {
        return tie(x, y) < tie(o.x, o.y);
    }
} point;

void swap(point p, point q){
    point r = p;
    p = q;
    q = r;
}

typedef struct arista_t {
    point p;
    point q;
    int64_t peso;

    bool operator==(const struct arista_t &o) const {
        return tie(p, q, peso) == tie(o.p, o.q, o.peso) or
               tie(p, q, peso) == tie(o.q, o.p, o.peso) ;
    }
    bool operator<(const struct arista_t &o) const {
        return tie(peso, p, q) < tie(o.peso, o.p, o.q);
    }
} arista;

vector<vector<list<point>>> componente_uf;
vector<vector<point>> padre_uf;

void init_uf(){
    padre_uf.resize(m);
    componente_uf.resize(m);
    for(int64_t y = 0; y < m ; y++) {
        padre_uf[y].resize(n);
        componente_uf[y].resize(n);
        for(int64_t x = 0; x < n ; x++){
            list<point> componente_nueva (1,{x,y});
            componente_uf[y][x] = componente_nueva;
            padre_uf[y][x] = {x,y};
        }
    }
}

point find_uf(point p){
    return padre_uf[p.y][p.x];
}

void union_uf(point p, point q){
    p = find_uf(p);
    q = find_uf(q);
    if (componente_uf[p.y][p.x].size() > componente_uf[q.y][q.x].size())
        swap(p,q);
    for(auto &r : componente_uf[p.y][p.x]){
        padre_uf[r.y][r.x] = q;
        componente_uf[q.y][q.x].push_back(r);
    }
    componente_uf[p.y][p.x].clear();
}

int64_t kruskal(){
    init_uf();
    set<arista> candidatos;
    int64_t res = 0;
    int64_t componentes_conexas = 0;

    //Agregar las aristas
    for (int64_t y = 1 ; y < m - 1 ; y++){
        for (int64_t x = 1 ; x < n - 1 ; x++){
            if (mapa[y][x] == 0){
                if (mapa[y][x+1] == 0)
                    candidatos.insert({{x,y}, {x+1,y}, 0});
                if (mapa[y+1][x] == 0)
                    candidatos.insert({{x,y}, {x,y+1}, 0});
                componentes_conexas++;
            }
            if (mapa[y][x] > 0){
                point p ={0,0} ,q;
                if(mapa[y-1][x] == 0) {
                    q = p;
                    p = {x,y-1};
                } 
                if(mapa[y+1][x] == 0) {
                    q = p;
                    p = {x,y+1};
                } 
                if(mapa[y][x-1] == 0) {
                    q = p;
                    p = {x-1,y};
                } 
                if(mapa[y][x+1] == 0) {
                    q = p;
                    p = {x+1,y};
                }
                candidatos.insert({p, q, mapa[y][x]});
            }
        }
    }

    //Hacer Kruskal
    for(auto &ar : candidatos) {
        if (find_uf(ar.p) == find_uf(ar.q))
            continue;
        union_uf(ar.p, ar.q);
        res += ar.peso;
        componentes_conexas--;

        if(componentes_conexas == 1)
            break;
    }

    return componentes_conexas == 1 ? res : -1;
}

\end{lstlisting}

\newpage
\subsection{Ejercicio 3}

\begin{lstlisting}[ basicstyle=\small]
#include "./include/interfaz.h"

#include <math.h>
#include <stdint.h>
#include <algorithm>
#include <cassert>
#include <iostream>
#include <map>
#include <queue>
#include <set>
#include <tuple>
#include <vector>

using namespace std;

typedef int64_t estacion;
typedef int64_t tiempo;

int64_t n;
int64_t generator = 0;
vector<vector<pair<estacion, tiempo>>> adj;
vector<estacion> padre;  // estacion -> <tiempo en llegar / predecesora>

int reventarDaistrah() {
    vector<bool> visitado = vector<bool>(n, false);
    vector<tiempo> tiempoMin(n, INT64_MAX);
    padre = vector<estacion>(n);

    tiempoMin[0] = 0;
    padre[0] = -1;

    while (true) {
        tiempo act_t = INT64_MAX;
        estacion est_t = -1;
        for (int j = 0; j < n; j++) {
            if (!visitado[j] and tiempoMin[j] < act_t) {
                act_t = tiempoMin[j];
                est_t = j;
            }
        }
        if (est_t == -1)
            break;

        visitado[est_t] = true;

        if (est_t == n - 1)
            return act_t;

        for (auto a : adj[est_t]) {
            estacion vecino = a.first;
            tiempo nuevo = act_t + a.second;
            tiempo anterior = tiempoMin[vecino];

            if (nuevo >= anterior)
                continue;

            tiempoMin[vecino] = nuevo;
            padre[vecino] = est_t;
        }
    }

    return -1;
}

\end{lstlisting}
