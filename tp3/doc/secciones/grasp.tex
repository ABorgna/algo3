\section{GRASP}
Nuestra última optimización se trata de una metaheurística \footnote{https://en.wikipedia.org/wiki/Metaheuristic} de tipo \emph{GRASP (Greedy randomized adaptive search procedure)}, que consiste en mejorar sucesivamente mediante búsqueda local una lista de soluciones candidatas arrojadas por nuestra función greedy aleatoria y luego tomando el mínimo de todas las soluciones ya optimizadas.
\\

Si bien el esquema de esta metaheurística es sencillo y concreto, hay mucho lugar para variaciones en la manera en que se toman las listas de candidatos (e.g tamaño, si se filtran con algún criterio, etc), criterios de parada o métodos de búsqueda local.
\\

\subsection{Búsquedas locales alternadas}
Nuestra primer combinación de variantes toma como parámetro un exponente $i$ que indica que se va a considerar un total de $(\#gimnasios + \#paradas)^i$ candidatos construídos por el \emph{greedy random}, otro exponente $j$ que indica que se van a realizar $(\#gimnasios + \#paradas)^j$ iteraciones de búsqueda local sobre los candidatos.
\\

Estas variables definen un criterio de parada que asegura un comportamiento polinómico sobre la cantidad de gimnasios y paradas totales.
\\

Además, para obtener mayor diversidad de características entre los candidatos, alternamos las búsquedas locales entre \emph{swap de nodos} y \emph{2opt} para optimizarlos.

\begin{lstlisting}
    $\textbf{Def}$ grasp(orden, i, j) $\rightarrow$ $\langle {float,\ int} \rangle$:
        mejorRes $\gets$ $\infty$

        n $\gets$ #gimnasios + #paradas

        cant_candidatos $\gets$ $n^{i}$
        limite $\gets$ $n^{j}$

        $\textbf{Para}$ i $\gets$ 0..cant_candidatos:
            resGreedy $\gets$ greedy_random(orden_actual)

            $\textbf{Si}$ resGreedy.first = $\infty$:
                $\textbf{Proximo i}$

            res $\gets$ local_dos_opt(orden_actual, limite) $\textbf{Si}$ i es par
                   local_swap(orden_actual, limite) $\textbf{Sino}$

            $\textbf{Si}$ res.first < mejorRes:
                mejorRes $\gets$ res.first
                orden $\gets$ orden_actual

        $\textbf{Retornar}$ $\langle$mejorRes, |orden|$\rangle$
\end{lstlisting}
