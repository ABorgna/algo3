\section{GRASP}
Nuestra última optimización se trata de una metaheurística \footnote{https://en.wikipedia.org/wiki/Metaheuristic} de tipo \emph{GRASP (Greedy randomized adaptive search procedure)}, que consiste en mejorar sucesivamente mediante búsqueda local una lista de soluciones candidatas arrojadas por nuestra función greedy aleatoria y luego tomando el mínimo de todas las soluciones ya optimizadas.
\\

Si bien el esquema de esta metaheurística es sencillo y concreto, hay mucho lugar para variaciones en la manera en que se toman las listas de candidatos (e.g tamaño, si se filtran con algún criterio, etc), criterios de parada o métodos de búsqueda local.
\\

\subsection{Búsquedas locales alternadas}
Nuestra primer combinación de variantes toma como parámetro un exponente $i$ que indica que se va a considerar un total de $(\#gimnasios + \#paradas)^i$ candidatos construídos por el \emph{greedy random}, otro exponente $j$ que indica que se van a realizar $(\#gimnasios + \#paradas)^j$ iteraciones de búsqueda local sobre los candidatos.
\\

Estas variables definen un criterio de parada que asegura un comportamiento polinómico sobre la cantidad de gimnasios y paradas totales.
\\

Además, para obtener mayor diversidad de características entre los candidatos, alternamos las búsquedas locales entre \emph{swap de nodos} y \emph{2opt} para optimizarlos.

\begin{lstlisting}
    $\textbf{Def}$ grasp_alternado(orden, i, j) $\rightarrow$ $\langle {float,\ int} \rangle$:
        mejorRes $\gets$ $\infty$

        n $\gets$ #gimnasios + #paradas

        cant_candidatos $\gets$ $n^{i}$
        limite $\gets$ $n^{j}$

        $\textbf{Para}$ i $\gets$ 0..cant_candidatos-1:
            resGreedy $\gets$ greedy_random(orden_actual)

            $\textbf{Si}$ resGreedy.first = $\infty$:
                $\textbf{Proximo i}$

            res $\gets$ local_dos_opt(orden_actual, limite) $\textbf{Si}$ i es par
                   local_swap(orden_actual, limite) $\textbf{Sino}$

            $\textbf{Si}$ res.first < mejorRes:
                mejorRes $\gets$ res.first
                orden $\gets$ orden_actual

        $\textbf{Retornar}$ $\langle$mejorRes, |orden|$\rangle$
\end{lstlisting}

\subsection{Mejores candidatos}
Mientras en el caso anterior nos permitíamos mejorar indiscriminadamente todas soluciones que nos devolvía el \emph{greedy random} una manera de reducir tiempos de ejecución es también ser más restrictivos sobre qué candidatos vamos a considerar y cuáles no.
\\

Todavía alternando búsquedas locales, presentamos uan segunda implementación de \emph{GRASP} para nuestro problema donde, todavía alternando búsquedas locales, solamente vamos a mejorar un grupo selecto de candidatos.
\\

En particular, nos limitamos únicamente a hacer búsquedas locales sobre a lo sumo la 'mejor mitad' (en cuanto a distancias de cada solución refiere) de las que anteriormente considerábamos. La intuición detrás de esto es que estas soluciones tendrían que iterar menos hasta encontrar un mínimo local o que den mejores resultados tras ser mejoradas.
\\

Para no tener que realizar ordenamientos, directamente agregamos los candidatos en un $set$ de tuplas ($distancia$, $orden$) que, al ordenar por primer componente, nos garantice acceso al mínimo y borrado logarítmico en sus elementos.

\begin{lstlisting}
    $\textbf{Def}$ grasp_trim(orden,  expLimite, expInicios) $\rightarrow$ $\langle$float, int$\rangle$
        mejorRes $\gets$ $\infty$

        n $\gets$ #gimnasios + #paradas

        cant_candidatos $\gets$ $n^{i}$
        limite $\gets$ $n^{j}$

        candidatos $\gets$ $\emptyset$
        $\textbf{Para}$ i $\gets$ 0...cant_candidatos-1:
            orden_candidato $\gets$ [ ]
            resGreedy $\gets$ greedy_random(orden_candidato)
            candidatos $\gets$ candidatos $\cup$ {$\langle$resGreedy.first, orden_candidato$\rangle$}

        hasta $\gets$ $\lceil \#candidatos / 2 \rceil$

        $\textbf{Para}$ i $\gets$ 0..hasta-1:
            actual $\gets$ min(candidatos)
            orden_actual $\gets$ actual.second
            candidatos $\gets$ candidatos - min(candidatos)

            $\textbf{Si}$ actual.first = $\infty$:
                $\textbf{Proximo i}$

                res $\gets$ local_dos_opt(orden_actual, limite) $\textbf{Si}$ i es par
                       local_swap(orden_actual, limite) $\textbf{Sino}$

                $\textbf{Si}$ res.first < mejorRes:
                    mejorRes $\gets$ res.first
                    orden $\gets$ orden_actual

        $\textbf{Retornar}$ $\langle$mejorRes, |orden|$\rangle$
\end{lstlisting}

\subsection{Con una única búsqueda local}
Por último hicimos dos implementaciones que, sin recortar candidatos, emplean una única búsqueda local (es decir, no alternan como en las implementaciones que ya mostramos) y mejoran cada instancia hasta encontrar un mínimo local.
\\

La motivación detrás de estas GRASPs es, a costa de potencialmente empeorar el rendimiento temporal (ya hemos visto que cuando $corridas=0$ para las búsquedas locales la performance es potencialmente factorial), mejorar la precisión dado que se mejoran todos los candidatos hasta que que no sea posible seguir.
\\

Para cada instancia hacemos un llamado a la respectiva búsqueda local de la función (i.e $2opt$ o $swap$) con $corridas=0$, recordando que este valor de parámetro seteaba la búsqueda para que realice iteraciones hasta no encontrar ningún vecino de la instancia final que presente menor distancia.

\begin{lstlisting}
    $\textbf{Def}$ grasp_local(orden, i) $\rightarrow$ $\langle {float,\ int} \rangle$:
        mejorRes $\gets$ $\infty$

        n $\gets$ #gimnasios + #paradas
        cant_candidatos $\gets$ $n^{i}$

        $\textbf{Para}$ i $\gets$ 0..cant_candidatos-1:
            resGreedy $\gets$ greedy_random(orden_actual)

            $\textbf{Si}$ resGreedy.first = $\infty$:
                $\textbf{Proximo i}$

            res $\gets$ busqueda_local(orden_actual, 0) $\emph{// 2opt/swap}$

            $\textbf{Si}$ res.first < mejorRes:
                mejorRes $\gets$ res.first
                orden $\gets$ orden_actual

        $\textbf{Retornar}$ $\langle$mejorRes, |orden|$\rangle$
\end{lstlisting}
