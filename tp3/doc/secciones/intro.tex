\section{Introducción}
    El problema del cual nos ocuparemos en este trabajo se presenta desde el siguiente contexto narrativo: queremos encontrar el tiempo mínimo para recorrer un conjunto de ubicaciones en un mapa (que llamaremos \emph{gimnasios}) pero manteniendo una cantidad no-negativa de ciertos objetos (\emph{pociones}) que disminuyen cada vez que recorremos un gimnasio. La cantidad de pociones presentes estará en todo momento acotada por una constante que consideraremos el \emph{tamaño de mochila}. Las pociones se recargan de a 3 unidades (si la recarga supera el tamaño de la mochila, el valor final será el tope mismo de la mochila) visitando ubicaciones que llamaremos \emph{paradas} y no pueden visitarse cada una más de una vez, sin embargo pueden no visitarse todas.
    \\

    Recibimos por input $n$ (parámetro) gimnasios de la forma $(X_k, Y_k, P_k)$ con las primeras y segundas coordenadas como su ubicación en el mapa (un plano euclideano) y las terceras como la cantidad de pociones que consumen. También recibimos $m$ (también parámetro) paradas de la forma $(X_{k'}, Y_{k'})$ que convertimos en tuplas $(X_{k'}, Y_{k'}, -3)$. Esto se debe a que vamos a considerar la secuencia de mínima distancia (definida más adelante), correspondiente al camino buscado, $orden = \langle {(X_0, Y_0, P_0)\ ..\ (X_{|orden|-1}, Y_{|orden|-1}, P_{|orden|-1})} \rangle$ que contenga solamente algunas paradas y todos los gimnasios (apareciendo una única vez cada uno) del input y definiremos la cantidad de pociones acumuladas al momento de realizar la i-ésima ubicación de la secuencia como:
    \\

    $$f(i)=  \left\{
            \begin{array}{lcc}
                -P_0                    &  si     & i = 0                \\
                \min(f(i-1) - P_i,\ $tamaño de mochila$) &  si     & 0 < i < |orden|          \\
            \end{array}
            \right.
   $$
   \\

   De modo que para todo $i$ valga $f(i) > 0$, manteniendo el requisito de la cantidad no-negativa de pociones acumuladas. Nos queda especificar la noción de distancia total de la secuencia:

   $$ distancia(orden) = \sum_{i=1}^{|orden|-1} \sqrt{(X_i - X_{i-1})^2 + (Y_i - Y_{i-1})^2} $$
   \\

   Por ejemplo, si se nos presenta una mochila con capacidad 4, dos gimnasios correspondientes a nodos $(0,0,2)$ y $(1,1,4)$ y dos paradas (0,1) y $(1,0)$, una solución óptima es aquella de distancia 3 dada por la secuencia $orden =\langle {(0,1), (0,0,2), (1,0), (1,1,4)} \rangle$ (no es la única óptima porque $orden = \langle {(1,0), (0,0,2), (0,1), (1,1,4)} \rangle$ también es solución con distancia 3).
   \\

   El problema es de la clase NP, esto se puede ver dado que se puede transformar una instancia del problema conocido como \emph{Shortest Hamiltonian Path} sobre grafos euclideanos (NP \footnote{Christos H. Papadimitriou, Computational Complexity, Page 190, Addison-Wesley Publishing Company, Inc.,1994.}) en una instancia de nuestro problema solamente tomando por cada nodo un gimnasio que consuma 3 pociones y una parada superpuesta al gimnasio. Permitiéndonos cierta informalidad, un supuesto algoritmo polinomial que resolvería nuestro problema recorrería de a pares superpuestos dado que, para mantener la cantidad de pociones no-negativa, hay que alternar paradas y gimnasios. Siendo claramente los gimnasios superpuestos a cada una de las paradas la forma óptima de tomar estos gimnasios, porque de lo contrario el camino estaría no solamente pagando la distancia entre gimnasios sino también entre gimnasios y paradas. Que sea de la clase NP implica que los algoritmos necesarios para conseguir una solución exacta al problema son del tipo de los que recorren el espacio de soluciones del problema buscando el orden apropiado para las ubicaciones. Mencionaremos más sobre esto en la sección correspondiente a dichas implementaciones.

   \subsection{Funciones auxiliares recurrentes y representación del grafo mapa}

   Antes de proceder específicamente sobre las implementaciones es conveniente explicar cómo representamos y tratamos las ubicaciones del mapa del problema. Si bien, como dijimos, las ubicaciones se corresponden a tuplas de la pinta $(X_k, Y_k, P_k)$, las enumeraremos como índices de 0 a $(n + m - 1)$ en el orden que se pasan por parámetro (primero $n$ gimnasios y luego $m$ paradas) de modo que para saber si una ubicación es gimnasio o parada basta con comparar su valor numérico contra $n$ y $m$.
   \\

   De esta manera, el grafo que sobre el cual trabajaremos consiste en dos secuencias de nodos $(X, Y, P)$ para gimnasios y paradas (el índice número $i$ es el nodo gimnasio i-ésimo o la parada (i-n)-ésima). Este grafo es completo dado que no hay restricción sobre qué ubicaciones se pueden visitar consecutivamente.
   \\

   También por conveniencia, presentamos el pseudocódigo de algunas funciones auxiliares que utilizaremos regularmente en las subsecciones de las implementaciones. Estas son aquellas que definen la validez de una secuencia como camino dado un tamaño de mochila y una cantidad acumulada de pociones hasta el momento (actualizando además dicha cantidad), y la distancia de un camino dado hasta su último gimnasio. Más adelante se detallará por qué, pero para buscar caminos mínimos en nuestro problema nos alcanza con saber distancias hasta el último gimnasio, sin contar las paradas restantes.
   \\

   \begin{lstlisting}
    $\textbf{Def}$ esCaminoValido(camino, mochila, pocionesAcum) $\rightarrow$ bool:
        $\textbf{Para}$ i en camino:
            $\emph{// Las paradas siempre valen -3}$
            nodo $\gets$ nodos_gimnasios[i] $\textbf{si}$ i < #Gimnasios $\textbf{sino}$ nodos_paradas[i - #Gimnasios]
            pocionesAcum $\gets$ pocionesAcum - nodo.p
            $\textbf{Si}$ pocionesAcum < 0:
                $\textbf{terminar ciclo}$
            $\textbf{Sino si}$ pocionesAcum > mochila:
                pocionesAcum $\gets$ mochila

        $\textbf{Retornar}$ pocionesAcum $\geq$ 0

    $\textbf{Def}$ esCaminoValido(camino, mochila) $\rightarrow$ bool:
        $\textbf{Retornar}$ esCaminoValido(camino, mochila, 0)
   \end{lstlisting}

   Es fácil ver que la complejidad es $\mathcal{O}(|camino|)$, dado que el cuerpo del ciclo solamente realiza comparaciones y asignaciones numéricas o de nodos (tuplas numéricas).

   \begin{lstlisting}
   $\textbf{Def}$ distanciaCamino(camino) $\rightarrow$ float:
       $\emph{//}$ |$\emph{camino}$| > 0
       dist $\gets$ 0
       ultimo $\gets$ camino[0]
       desdeUltimoGim $\gets$ 0
       $\textbf{Para}$ i en camino:
           desdeUltimoGim $\gets$ desdeUltimoGim + distancia(ultimo, i)
           $\textbf{Si}$ i < #Gimnasios:
               dist $\gets$ dist desdeUltimoGim
               desdeUltimoGim $\gets$ 0
           ultimo $\gets$ i

       $\textbf{Retornar}$ dist

    $\textbf{Def}$ distancia(i, j) $\rightarrow$ float:
        nodo_i $\gets$ nodos_gimnasios[i] $\textbf{si}$ i < #Gimnasios $\textbf{sino}$ nodos_paradas[i - #Gimnasios]
        nodo_j $\gets$ nodos_gimnasios[j] $\textbf{si}$ j < #Gimnasios $\textbf{sino}$ nodos_paradas[j - #Gimnasios]
        dx $\gets$ nodo_i.x - nodo_i.x
        dy $\gets$ nodo_i.y - nodo_i.y
        $\textbf{Retornar}$ $sqrt(dx^2 + dy^2)$
   \end{lstlisting}

   Para $distanciaCamino$ se vuelve a proponer una complejidad temporal peor caso de $\mathcal{O}(|camino|)$ dado que las asignaciones y comparaciones del resto del algoritmo son $\mathcal{O}(1)$ (se asume esta complejidad para la operación $distancia$ si consideramos $sqrt$ constante en peor caso independientemente de la arquitectura, considerando una cantidad de $ticks$ de CPU necesaria acotada \footnote{http://stackoverflow.com/questions/6884359/c-practical-computational-complexity-of-cmath-sqrt})




\subsection{Generadores}

Para poder probar nuestras heurísticas diseñamos una serie de generadores de instancias aleatorios que se detalla a continuación.

Cada uno recibe como variable la cantidad de gimnasios y paradas, y el tamaño de la mochila deseados
(y opcionalmente la seed a utilizar en el generador de números aleatorios).

\subsubsection{Generador \texttt{random}}
\label{sec:generadores-random}

El generador mas general, que usaremos en la mayoría de las mediciones.
Es capaz de generar cualquier instancia válida
(con las posiciones $x$ e $y$ dentro de los límites dados).

Comienza ubicando cada uno de los gimnasios y paradas en una posición aleatoria con $0 \leq x,y \leq 65535$.
A cada uno de gimnasios le asigna una cantidad de pociones aleatoria entre $0$ y $10$, cuidando que la cantidad total nunca supere 3 veces la cantidad de paradas.

A continuación puede verse el pseudocódigo:

\begin{lstlisting}
    $\textbf{Def}$ randomGenerator(int nGyms, int nStops, int bagSize) $\rightarrow$ {[Gimnasio], [Parada]}:
        gyms $\gets$ {}
        stops $\gets$ {}
        pocionesRestantes $\gets$ nStops * 3

        $\textbf{Para}$ i de 0 a nGyms:
            int gymsRestantes $\gets$ nGyms - i - 1
            int potas $\gets$ min(random(0,10), pocionesRestantes - gymsRestantes)
            potas = min(potas, bagSize)

            int x $\gets$ random(0,65535)
            int y $\gets$ random(0,65535)
            agregar a gyms un gimnasio con {x,y,potas}
            pocionesRestantes $\gets$ pocionesRestantes - potas

        $\textbf{Para}$ i de 0 a nStops:
            int x $\gets$ random(0,65535)
            int y $\gets$ random(0,65535)
            agregar a stops una parada con {x,y}

        $\textbf{Retornar}$ {gyms, stops}
\end{lstlisting}

Es importante notar que existen casos donde se podría generar instancias sin solución.
Por ejemplo, si deseamos una instancia con 2 gimnasios, 3 paradas y tamaño de mochila 6
podría generarse el problema descripto en el cuadro \ref{tab:generadores-random-sinsolu}.

\begin{table}[H]
    \begin{center}
        \begin{tabular}{l | c c c}
            Nodo   & X & Y & potas \\
            \hline
            Gimnasio 1 & 0 & 0 &  5 \\
            Gimnasio 2 & 1 & 0 &  4 \\
            Parada 1   & 2 & 0 & -3 \\
            Parada 2   & 3 & 0 & -3 \\
            Parada 3   & 4 & 0 & -3 \\
        \end{tabular}
        \caption{Instancia sin solución}\label{tab:generadores-random-sinsolu}
    \end{center}
\end{table}

\subsubsection{Generador \texttt{separated}}

La idea de este generador es agrupar los gimnasios y las paradas separando unos de otros,
y generando dos \textit{cúmulos} de puntos.

El algoritmo es similar al del \texttt{random}, descripto en la sección \ref{sec:generadores-random}, pero restringiendo las posiciones de los gimnasios a $0 \leq x,y \leq 16383$, y la de las paradas a $0 \leq y \leq 16383$ y $32768 \leq x \leq 49152$.

\subsubsection{Generador \texttt{zigzag}}

Este generador intenta generar instancias donde el camino solución deba ir zigzageando entre dos columnas. Para ello fuerza el tamaño de la mochila a 3, la cantidad de pociones a 3, y coloca los gimnasios y las paradas en dos columnas separadas.

El algoritmo nuevamente es una modificación del \texttt{random} de la sección \ref{sec:generadores-random},
esta vez fijando $x = 0$ para los gimnasios e $x = 16383$ para las paradas y variando la coordenada $y$ de ambos entre $0$ y $16383$.
Además, setea el tamaño de la mochila a $3$ y la cantidad de pociones de cada gimnasio a $3$ (si hay suficientes paradas, sino debe dejar algunos en 0 para asegurar que haya solución).

Este nuevo pseudocódigo se muestra a continuación:

\begin{lstlisting}
    $\textbf{Def}$ zigzagGenerator(int nGyms, int nStops, int bagSize) $\rightarrow$ {[Gimnasio], [Parada]}:
        gyms $\gets$ {}
        stops $\gets$ {}
        setear el tamano de la mochila a 3

        $\textbf{Para}$ i de 0 a nGyms:
            int potas $\gets$ 3 if i < nStops else 0
            int y $\gets$ random(0,16383)
            agregar a gyms un gimnasio con {0,y,potas}

        $\textbf{Para}$ i de 0 a nStops:
            int y $\gets$ random(0,16383)
            agregar a stops una parada con {16383,y}

        $\textbf{Retornar}$ {gyms, stops}
\end{lstlisting}
