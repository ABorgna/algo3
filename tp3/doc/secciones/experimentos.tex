\section{Experimentación}

Para las diferentes mediciones necesarias para los experimentos presentados a continuación fueron realizadas sobre una CPU Intel 5200U en un sistema con 8GB de memoria RAM. Los binarios fueron compilados por \texttt{gcc 6.2.1 20160830} utilizando los flags \texttt{-O3 -std=c++11 -march=native}. El toolchain utilizado para correr las mediciones y realizar los gráficos puede encontrarse junto al código en \texttt{tools/data_analisis}.

Para las mediciones de tiempos se corre repetidamente el mismo problema no menos de 4 veces y 100ms para caso, y luego se almacena la media de las mediciones.

\subsection{Runtime de los algoritmos exactos}

\subsubsection{En función de la cantidad de nodos}

En la figura \ref{fig:time-exacto} comparamos el tiempo de ejecución del algoritmo exacto de fuerza bruta con el de backtracking.
Para ello medimos con ambos 5 casos distintos para cada combinación de cantidad de gimnasios y paradas entre 2 y 5. Además usamos siempre un tamaño de mochila mayor a tres veces la cantidad de paradas.

Como podemos observar, ambos se comportan exponencialmente pero el backtracking resulta ser cerca de un orden de magnitud mas rápido.

\begin{figure}[H]
	\centering
	\includegraphics[width=14cm]{time-exacto}
	\caption{Tiempo de ejecución de los algoritmos exactos}
	\label{fig:time-exacto}
\end{figure}

\subsubsection{En función del tamaño de la mochila}

A continuación realizamos pruebas fijando la cantidad de gimnasios y paradas en 5 y variando el tamaño de la mochila.

Desafortunadamente vemos en la figura \ref{fig:time-exacto-moch} que dentro del intervalo de tamaños de mochila
interesantes, limitado por la cantidad de nodos que puede tener un problema a ser resuelto en un tiempo factible
(ya que si el tamaño es mayor a tres veces la cantidad de paradas, no influirá en el comportamiento del algoritmo),
no podemos determinar propiamente el comportamiento.

\begin{figure}[H]
	\centering
	\includegraphics[width=13.5cm]{time-exacto-moch}
	\caption{Tiempo de ejecución de los algoritmos exactos al variar el tamaño de la mochila}
	\label{fig:time-exacto-moch}
\end{figure}

\subsection{Runtime de las heurísticas greedy}

\subsubsection{En función de la cantidad de nodos}

Medimos el tiempo en la figura \ref{fig:time-greedy} generando 5 casos para cada combinación de entre 1000 y 10000 gimnasios y paradas, avanzando de a 1000, y tamaño de mochila mayor a 30000.

Vemos en la figura \ref{fig:time-greedy-correlation} que estos tiempos tiempos se correlaciona fuertemente con un comportamiento cuadrático en función de la cantidad de nodos.

\begin{figure}[H]
	\centering
	\includegraphics[width=13.5cm]{time-greedy}
	\caption{Tiempo de ejecución de las heurísticas greedy}
	\label{fig:time-greedy}
\end{figure}

\begin{figure}[H]
	\centering
	\includegraphics[width=10cm]{time-greedy-correlation}
	\caption{Correlación entre el tiempo de ejecución del greedy y una complejidad cuadrática}
	\label{fig:time-greedy-correlation}
\end{figure}

\subsubsection{En función del tamaño de la mochila}

Realizamos también pruebas fijando la cantidad de gimnasios y paradas en 2500 y variando el tamaño de la mochila entre 100 y 7500 en pasos de a 100, y nos encontramos con un comportamiento constante y con poco ruido para ambos tipos de greedy (con una correlación entre tiempo medido y tamaño de mochila de $0.170856$).

En el cuadro \ref{tab:time-greedy-moch} se encuentra una descripción de los datos medidos.

\begin{table}[H]
    \begin{center}
        \begin{tabular}{| l r |}
            \hline
            count  & 76.000000 \\
            mean   &  0.215175 \\
            std    &  0.005658 \\
            min    &  0.209863 \\
            25\%   &  0.212328 \\
            50\%   &  0.213131 \\
            75\%   &  0.215283 \\
            max    &  0.239380 \\
            \hline
        \end{tabular}
        \caption{Descripción de las mediciones en función del tamaño de mochila}\label{tab:time-greedy-moch}
    \end{center}
\end{table}

\subsection{Runtime de las heurísticas locales}

\subsubsection{En función de la cantidad de nodos}

Para las heurísticas locales medimos nuevamente los tiempos generando 5 casos para cada combinación de entre 10 y 100 gimnasios y paradas, avanzando de a 10, y tamaño de mochila mayor a 300. Los resultados se pueden apreciar en la figura \ref{fig:time-local}.

Podemos corroborar en la figura \ref{fig:time-local-correlation} que ambas variaciones se corresponden con un comportamiento de orden cuarto en función de la cantidad de nodos.

\begin{figure}[H]
	\centering
	\includegraphics[width=13.5cm]{time-local}
	\caption{Tiempo de ejecución de las heurísticas locales}
	\label{fig:time-local}
\end{figure}

\begin{figure}[H]
	\centering
	\includegraphics[width=13cm]{time-local-correlation}
    \caption{Correlación entre el tiempo de ejecución de las heurísticas locales 2opt (izquierda) y swap (derecha), con una complejidad $n^4$}
	\label{fig:time-local-correlation}
\end{figure}

\subsubsection{En función del tamaño de la mochila}

Nuevamente realizamos pruebas fijando la cantidad de gimnasios y paradas en 50 y variando el tamaño de la mochila entre 10 y 150, y nos encontramos, al igual que con el greedy, con un comportamiento constante en función de la mochila para ambas variaciones (con una correlación entre tiempo medido y tamaño de mochila de $0.259633$ para swap y de $0.265672$ para 2opt).

En el cuadro \ref{tab:time-local-moch} se encuentra una descripción de los datos medidos.

\begin{table}[H]
    \begin{center}
        \begin{tabular}{| l r r |}
            \hline
            & Swap & 2opt \\
            \hline
            count  & 141.000000 & 141.000000 \\
            mean   &   0.263016 &   0.167931 \\
            std    &   0.080163 &   0.034468 \\
            min    &   0.124284 &   0.087849 \\
            25\%   &   0.214029 &   0.143264 \\
            50\%   &   0.253680 &   0.165566 \\
            75\%   &   0.297979 &   0.193758 \\
            max    &   0.566984 &   0.268200 \\
            \hline
        \end{tabular}
        \caption{Descripción de las mediciones en función del tamaño de mochila}\label{tab:time-local-moch}
    \end{center}
\end{table}

\subsection{Runtime de la metaheurística GRASP}

\subsubsection{En función de la cantidad de nodos}

Para medir el comportamiento de GRASP generamos 4 instancias para cada combinación de gimnasios y mochilas entre 10 y 50, avanzando de a 5, con un tamaño de mochila superior a 150. En la figura \ref{fig:time-grasp} pueden apreciarse los resultados. Como se observa en la figura \ref{fig:time-grasp-correlation}, el comportamiento está fuertemente correlacionado a un orden 5 sobre la cantidad de nodos.

\begin{figure}[H]
	\centering
	\includegraphics[width=13.5cm]{time-grasp}
	\caption{Tiempo de ejecución de la metaheurística grasp}
	\label{fig:time-grasp}
\end{figure}

\begin{figure}[H]
	\centering
	\includegraphics[width=10cm]{time-grasp-correlation}
	\caption{Correlación entre el tiempo de ejecución de grasp y una complejidad de orden 5 en función de la cantidad de nodos}
	\label{fig:time-grasp-correlation}
\end{figure}

\subsubsection{En función del tamaño de la mochila}

Para medir el tiempo de ejecución en función del tamaño de la mochila fijamos la cantidad de gimnasios y paradas en 20 y variamos la mochila entre 10 y 60. Como se aprecia en la figura \ref{fig:time-grasp-moch}, nos encontramos con que para tamaños pequeños se observa una correlación entre la mochila y el tiempo de ejecución, pero mas o menos a partir del tamaño 30 los tiempos se mantienen relativamente constantes.

Esto puede deberse a que cuando hay poco espacio el la mochila se limita la cantidad de opciones que puede tomar el algoritmo, lo que hace que termine mas temprano.

\begin{figure}[H]
	\centering
	\includegraphics[width=10cm]{time-grasp-moch}
	\caption{Correlación entre el tiempo de ejecución de grasp y una complejidad de orden 5 en función de la cantidad de nodos}
	\label{fig:time-grasp-moch}
\end{figure}
