\section{Heurística constructiva-golosa}
    Como vimos en la sección anterior, el costo en complejidad temporal de un algoritmo que nos asegure una solución exacta a nuestro problema no es desdeñable incluso aplicando podas y estrategias para aminorar tiempos de ejecución. En esta sección nos encargaremos de presentar técnicas heurísticas para el problema, particularmente del tipo \emph{greedy}, que proporcionen soluciones en tiempo polinomial aún sin proveer garantías de soluciones óptimas.

\subsection{Gimnasio más cercano}
    Ya mencionamos en la introducción cierto paralelismo en la naturaleza del problema con el de buscar un camino hamiltoniano mínimo, el cual a su se asemeja al \emph{TSP}\footnote{Travelling Salesman Problem: https://en.wikipedia.org/wiki/Travelling_salesman_problem}. Nuestro \emph{approach} a una heurística \emph{greedy} está fuertemente basada en el homólogo \emph{Nearest neighbour algorithm} \footnote{https://en.wikipedia.org/wiki/Nearest_neighbour_algorithm} pensada para encontrar soluciones al \emph{TSP}.
    \\

    La idea es tomar en cada momento, de todos los gimnasios con requisitos de pociones menores a la cantidad acumulada, aquel que sea más cercano a la ubicación 'actual'. De no contar con suficientes pociones para ninguno de los gimnasios restantes, se busca la parada más cercana. En caso de que no queden paradas sin visitar y falten gimnasios por visitar el algoritmo terminará, devolviendo que no hay solución.
    \\

    En muchas ocasiones el algoritmo puede no encontrar una solución factible incluso frente a la existencia de una. Particularmente por la 'saturación' de pociones respecto del tamaño de la mochila, tomando menos pociones de las que ofrece una parada (no pudiendo cumplir eventualmente el requisito de algún gimnasio posterior). Por ejemplo, sea un tamaño de mochila 5 y los siguientes 5 gimnasios y paradas:
    \\

    (2, 2, 4)     \quad  \quad  (0, 0)

    (2, 5, 5)      \quad \quad  (1, 2)

    (10, 10, 4)    \quad   (2, 3)

    (0, 1, 1)    \quad  \quad  (2, 4)

    (1, 1, 1)    \quad  \quad  (3, 5)
    \\

    El algoritmo \emph{exacto} toma la secuencia $\langle$(0, 0), (0, 1, 1), (2, 4), (2, 5, 5), (2, 3), (1, 1, 1), (1, 2), (2, 2, 4), (3, 5), (10, 10, 4)$\rangle$ con distancia aproximada $23,60$. En cambio, el algoritmo \emph{greedy} comienza por la primer parada enumerada (como se verá más adelante en el pseudocódigo) $(0, 0)$ dado que no puede acceder a ningún gimnasio por requisitos de pociones, para luego ir a los gimnasios más cercanos $(0, 1, 1)$ y $(1, 1, 1)$ quedando con una única poción. En este punto accede a la parada $(1, 2)$ para luego visitar (con 4 pociones acumuladas) el gimnasio $(2, 2, 4)$. Como ahora los dos únicos gimnasios que quedan requieren más de 3 pociones tiene que volver a dos paradas, pero al tener una mochila de tamaño 5 va a saturar las 3 potenciales pociones en 5 necesariamente. Yendo al gimnasio más cercano $(2,5,5)$ queda sin pociones y con una única parada restante por visitar y un gimnasio de 4 pociones por requisito. Por lo tanto no puede terminar el camino.
    \\

    Incluso no solo no provee garantías de llegar a un resultado válido, de existir, sino que tampoco podemos especular con un 'error máximo' de precisión respecto de la solución real. Supongamos que existe alguna cota k que caracterice "cuántas veces la distancia mínima real" puede a lo sumo llegar a ser aquella devuelta por nuestro algoritmo. Pero entonces podríamos suponer un escenario con tamaño de mochila $3*k$, $n$ paradas (0,0) superpuestas y $n$ gimnasios ($k$, 0, 3) también superpuestos.
    \\

    El algoritmo \emph{exacto} devolvería un orden que tome primero todas las paradas recorriendo distancia 0 y luego todos los gimnasios, con distancia $k$.
    \\

    El \emph{greedy}, en cambio, va a buscar la primer parada para tener al menos 3 pociones (necesarias para visitar cualquier gimnasio) para luego visitar alguno. Al quedar sin pociones, va a tener que reiterar el mecanismo repetidas veces hasta cubrir todas las paradas y gimnasios de manera alternada, haciendo $2k$ idas y $2k-1$ vueltas. Por lo tanto va a recorrer $k(2k-1)$ que es $2k-1$ veces la distancia del exacto, superando la supuesta cota $k$.

    \subsubsection{Pseudocódigo del algoritmo}

    Haciendo uso de conjuntos, implementados sobre estructuras de acceso logarítmico en su cardinal \footnote{http://en.cppreference.com/w/cpp/container/set}, guardamos por un lado el total de nodos paradas (numerados desde $\#gimnasios$ hasta $(\#gimnasios + \#paradas - 1)$ y, por el otro lado, por cada gimnasio (van de 0 a $\#gimnasios-1$) una tupla de $\langle pociones\ requeridas,\ gimnasio \rangle$ de modo que tengamos acceso logarítmico al gimnasio con menos pociones requeridas (el tipo set ordena primero por primer componente en pares). Esto nos va a permitir recortar búsquedas lineales cuando, al recorrer de menor a mayor en requerimientos de pociones, veamos el primer gimnasio que supera nuestra cantidad acumulada.
    \\

    Como el esquema del vecino más cercano nos permite elegir cualquier ubicación inicial válida, tomamos como punto de partida del camino el gimnasio 'gratuito' de menor numeración (correspondiente al mínimo sobre el conjunto, dado que es el elemento con menor segunda componente de aquellos con pociones requeridas igual a 0), sino, la primer parada enumerada.
        \begin{lstlisting}
        distRecorrida $\gets$ 0
        orden $\gets$ $\langle {\ } \rangle$

        gims $\gets$ $\emptyset$
        $\textbf{Para}$ i $\gets$ 0.. #gimnasios:
            gims $\gets$ gims $\cup\ \{ \langle P(i),\ i \rangle \}$

        paradas $\gets$ $\emptyset$
        $\textbf{Para}$ i $\gets$ 0.. #paradas:
            paradas $\gets$ paradas $\cup\ \{i + \#gimnasios\}$
        pociones_actuales $\gets$ 0

        $\textbf{Si}$ min(gims).first > 0:
            nodo_actual $\gets$ min(paradas)
            paradas $\gets$ paradas - min(paradas)
            pociones_actuales $\gets$ min{3, tam_mochila}
        $\textbf{Sino:}$
            nodo_actual $\gets$ min(gims).second
            gims $\gets$ gims - min(gims)
        orden $\gets$ orden ++ [nodo_actual]

        Mientras gims $\neq$ $\emptyset$:
            $\textbf{Si}$ min(gims).first $\leq$ pociones_actuales:
                gim_candidato $\gets$ $(g \in$ gims | g.first $ \leq $ pociones_actuales $\land$
                  ($\forall \ g'\in$ gims, g'.first$\leq$ pociones_actuales) $distancia($nodo_actual, g.second$) \leq$
                                                            $distancia($nodo_actual, g'.second))

                dist_candidato $\gets$ $distancia$(nodo_actual, gim_candidato.second)

                distRecorrida $\gets$ distRecorrida + dist_candidato
                nodo_actual $\gets$ gim_candidato.second
                pociones_actuales $\gets$ pociones_actuales - gim_candidato.first
                gims $\gets$ gims - {gim_candidato}

            $\textbf{Sino si}$ paradas $\neq$ $\emptyset$:
                parada_mas_cercana $\gets$ $(p \in$ paradas | $\forall \ p'\in$ paradas, $distancia($nodo_actual$, p) \leq$
                                                                 $distancia($nodo_actual$, p'))$

                dist_candidato $\gets$ $distancia$(nodo_actual, parada_mas_cercana)

                distRecorrida $\gets$ distRecorrida + dist_candidato
                nodo_actual $\gets$ parada_mas_cercana
                pociones_actuales $\gets$ min(pociones_actuales + 3, tam_mochila)
                paradas $\gets$ paradas - {nodo_actual}

            $\textbf{Sino:}$
                distRecorrida $\gets$ -1
                $\textbf{Terminar}$
            orden $\gets$ orden ++ [nodo_actual]
        $\textbf{Retornar}$ {distRecorrida, orden.size()}
        \end{lstlisting}

\subsubsection{Complejidad del algoritmo}
    El costo de inicializar ambos conjuntos con todos sus elementos iniciales es $\mathcal{O}(\#gimnasios * log(\#gimnasios)\ +\ \#paradas*log(\#paradas))$ acotando para ambos casos el costo de cada inserción como aquel en el que ya se encuentran todos los elementos en la estructura. Además, conseguir los requisitos de pociones de cada gimnasio es $\mathcal{O}(1)$ considerando que esa información está contenida en las tuplas que representan al grafo, mencionadas en la sección de introducción, las cuales se coleccionan en vectores de acceso $\mathcal{O}(1)$ desde índices.
    \\

    Para determinar la posición inicial hace falta consultar la cantidad de pociones que pide el gimnasio de menor requisitos, esto tiene un costo asociado de $\mathcal{O}(log(\#gimnasios))$. Si no es 'gratuito' entonces tenemos que asignar la ubicación actual como la menor parada numerda y eliminarla del conjunto de paradas candidatas, ambas operaciones en $\mathcal{O}(log(\#paradas))$, y luego incrementar la cantidad acumulada de pociones tomando mínimo entre las 3 que otorga la parada y el tamaño total de mochila en $\mathcal{O}(1)$. De ser gratuito, se asigna el mínimo gimnasio como ubicación actual y se lo elimina de la lista de candidatos en $\mathcal{O}(log(\#gimnasios))$. En ambos casos, agregar la nueva ubicación a la lista de orden es $\mathcal{O}(1)$ amortizado, implementado en el tipo vector de la STL de C++ \footnote{http://www.cplusplus.com/reference/vector/vector/push_back/\#complexity}. Por lo tanto asignar la primer ubicación tiene un costo $\mathcal{O}(log(\#gimnasios)+log(\#paradas))$ sumando ambas ramas y la guarda.
    \\

    Nos resta ver el costo del ciclo. El ciclo itera hasta cubrir todos los gimnasios candidatos, pero en peor caso necesita para ello cubrir también todas las paradas. Entonces el ciclo itera, en peor caso, un orden de $(\#gimnasios+\#paradas)$ veces.
    \\

    En cada iteración se consulta si el conjunto de gimnasios candidatos es vacío, en tiempo constante \footnote{http://www.cplusplus.com/reference/vector/vector/empty/\#complexity}. Y además se inserta, como vimos ya en $\mathcal{O}(1)$ amortizado, la ubicación elegida.
    \\

    En la rama en que no alcanzan las pociones actuales y quedan paradas hace falta recorrer linealmente todas las paradas en busca de aquella más cercana, como la operación de distancia tiene complejidad $\mathcal{O}(1)$ (visto en la introducción) el costo de buscar la más cercana tiene en el peor caso asociado a recorrer todas las paradas sin considerar borrados de $\mathcal{O}(\#paradas)$. Luego se actualizan variables numéricas como pociones actuales y distancia recorrida en $\mathcal{O}(1)$ y finalmente se elimina la parada elegida en $\mathcal{O}(log(\#paradas))$. La rama tiene un costo peor caso, entonces, de $\mathcal{O}(\#paradas)$ considerando también el costo de consultar en la guarda si el conjunto de candidatos está vacío que, como ya vimos, es constante.
    \\

    En la rama en que existe al menos un gimnasio recorrible con la cantidad de pociones acumuladas, también hace falta buscar linealmente en la lista de gimnasios a aquel con menor distancia y que además pida menos pociones que las acumuladas, estas dos operaciones se realizan en tiempo constante y por lo tanto la búsqueda tiene un costo peor caso (al igual que en las paradas, considerando que siempre van a estar todos los gimnasios sin considerar que se van eliminando) de $\mathcal{O}(\#gimnasios)$. Nuevamente se actualizan las pociones, distancias recorridas, y ubicación actual en tiempo constante para finalmente eliminar el gimnasio elegido del conjunto de candidatos en $\mathcal{O}(log(\#gimnasios))$. Sumando todos estos costos al de consultar en la guarda si el gimnasio de requisito mínimo pide menos pociones de las que tenemos en $\mathcal{O}(log(\#gimnasios))$ peor caso, nos queda una complejidad total de la rama de $\mathcal{O}(\#gimnasios)$.
    \\

    Despreciando el costo de la rama en que se aborta la ejecución porque solamente se actualiza la distancia recorrida y se retorna. Nos queda un costo total del ciclo que ejecuta $(\#gimnasios+\#paradas)$ con un costo por cuerpo de $(\#gimnasios+\#paradas)$, quedando la complejidad $\mathcal{O}((\#gimnasios+\#paradas) * (\#gimnasios+\#paradas)) = \mathcal{O}((\#gimnasios+\#paradas)^2)$.

    \subsection{Gimnasio más lejano}
    Como nos interesaba poder comparar el \emph{greedy} anterior con otra heurística constructiva golosa a la hora de contrastar rendimientos con otras, implementamos otra heurística similar. El código es idéntico al de 'Gimnasio más cercano' salvo que en cada búsqueda lineal del siguiente candidato, se elije aquel que maximice la distancia al nodo actual.
    \\

    El algoritmo está basado en el esquema de inserción más lejana. La idea es que, como en muchos grafos hay ubicaciones costosas de recorrer, estas sean visitadas cuanto antes con la esperanza de que las remanentes formen un grafo más compacto.
    \\

    La complejidad es exactamente la misma y la implementación solo cambia la comparación de menor estricto a mayor estricto para las distancias de los candidatos.


    \subsection{Elección aleatoria sobre un subconjunto de gimnasios}

    La estrategia de este algoritmo es similar a la del gimnasio más cercano. Su diferencia se basa en la elección de gimnasios y paradas. En vez de elegir el nodo más cercano a la ubicación 'actual', se elige uno aleatoriamente entre la mitad de los nodos más cercanos a la misma ubicación.
    \\

    La lógica de elección entre ir a un gimnasio o a una parada es la misma: si no se puede ir a ningún gimnasio, entonces se busca una parada con el mismo criterio mencionado en el párrafo anterior, y luego se continua con el procedimiento desde la nueva posición 'actual'.
    \\

    Al mismo tiempo, así como el algoritmo anterior, este también puede no encontrar una solución válida aún en el caso de haberla.
    \\

    La estrategia de este algoritmo se basa en aumentar las posibilidades de elecciones al tomar uno aleatoriamente entre la mitad de los más cercanos, ya que existen situaciones en las cuales tomar el más cercano de todos no es la mejor solución.
    \\

    La sección que cambia de manera más significativa el código es en las guardas a la hora de buscar gimnasios o paradas.
    \\

        \begin{lstlisting}
            $\textbf{Si}$ min(gims).first $\leq$ pociones_actuales:
                gims_candidatos $\gets$ vector de gimnasios que se pueden visitar

                sort(gims_candidatos)           //Se ordenan segun la distancia al nodo actual

                gim_candidato $\gets$ random(primera_mitad(gims_candidatos))

                [...]

            $\textbf{Sino si}$ paradas $\neq$ $\emptyset$:
                paradas_mas_cercanas $\gets$ vector de todas las paradas

                sort(paradas_mas_cercanas)      //Se ordenan segun la distancia al nodo actual

                parada_mas_cercana $\gets$ random(primera_mitad(paradas_mas_cercanas))

                [...]

        \end{lstlisting}

    El primer paso en cada entrada de la guarda es recorrer de manera lineal tanto los gimnasios como las paradas, con lo cual la complejidad se mantiene igual. El segundo paso es ordenar. Esto cambia la complejidad, ya que anteriormente se recorrían linealmente los gimnasios o las paradas respectivamente dependiendo de en cual de las dos condiciones de la guarda se ejecutaba. Ahora, al realizar un sort, esa parte del algoritmo pasa de ser $\mathcal{O}(\#gimnasios + \#paradas)$ a ser $\mathcal{O}(\#gimnasios \cdot log(\#gimnasios) + \#paradas \cdot log(\#paradas))$. Obtener un elemento random de un arreglo tiene complejidad $O(1)$ ya que se puede implementar con la función rand que provee C++\footnote{http://stackoverflow.com/questions/8658784/rand-function-in-c}, cuya complejidad es la mencionada aquí. Siguiendo la linea de razonamiento para el algoritmo del gimnasio más cercano y aplicando esta modificación, la complejidad nueva pasa a ser $\mathcal{O}((\#gimnasios + \#paradas) \cdot (\#gimnasios \cdot log(\#gimnasios) + \#paradas \cdot log(\#paradas)))$