\section{Códigos Fuente}
\subsection{Ejercicio 1}

\lstset{language=C++,
                basicstyle=\ttfamily,
                keywordstyle=\color{blue}\ttfamily,
                stringstyle=\color{red}\ttfamily,
                commentstyle=\color{green}\ttfamily,
                morecomment=[l][\color{magenta}]{\#}
}

\begin{lstlisting}[ basicstyle=\small]
    #include <math.h>
    #include <stdint.h>
    #include <algorithm>
    #include <cassert>
    #include <iostream>
    #include <map>
    #include <queue>
    #include <set>
    #include <tuple>
    #include <vector>

    using namespace std;

    // Hay 2^(N+M) estados posibles
    typedef struct estado_t {
        // Para todo, true = del lado derecho
        vector<bool> arq;
        vector<bool> can;
        bool linterna = false;

        estado_t(){};
        estado_t(uint32_t n, uint32_t m) : arq(n), can(m) {}

        // La priority_queue requiere <
        bool operator<(const struct estado_t &o) const {
            return tie(linterna, arq, can) < tie(o.linterna, o.arq, o.can);
        }
        bool operator==(const struct estado_t &o) const {
            return tie(linterna, arq, can) == tie(o.linterna, o.arq, o.can);
        }
    } estado;

    vector<uint32_t> arq;
    vector<uint32_t> can;

    bool agregarCandidato(map<estado, uint32_t> &dist,
                          set<pair<uint32_t, estado>> &candidatos, estado &hijo,
                          uint32_t peso);

    int64_t tiempoMinimo(const vector<uint32_t> &arq, const vector<uint32_t> &can) {
        // Empiezan todos del lado izquierdo
        estado inicial(arq.size(), can.size());
        estado objetivo;
        objetivo.arq = vector<bool>(arq.size(), true);
        objetivo.can = vector<bool>(can.size(), true);
        objetivo.linterna = true;

        // Distancia minima hasta el estado
        map<estado, uint32_t> dist;
        dist.insert({inicial, 0});

        // Mantenemos la lista de candidatos a explorar,
        // y vemos siempre el mas cercano primero
        set<pair<uint32_t, estado>> candidatos;
        candidatos.insert(make_pair(0, inicial));

        int64_t res = -1;
        while (!candidatos.empty()) {
            auto p = candidatos.begin();
            uint32_t costo = p->first;
            estado nodo = p->second;
            candidatos.erase(p);

            // Llegamos a la solucion
            if (nodo == objetivo) {
                res = costo;
                break;
            }

            // Hacemos un population count
            uint32_t arqIzq = 0;
            uint32_t canIzq = 0;

            for (uint32_t i = 0; i < arq.size(); i++)
                if (nodo.arq[i])
                    arqIzq++;
            for (uint32_t i = 0; i < can.size(); i++)
                if (nodo.can[i])
                    canIzq++;

            // Checkeamos que no sea un caso invalido
            if ((arq.size() - arqIzq > 0) &&
                (arq.size() - arqIzq) < (can.size() - canIzq)) {
                continue;
            }
            if (arqIzq > 0 && arqIzq < canIzq) {
                continue;
            }

            // Generamos los pasos posibles a partir de donde estamos
            estado hijo = nodo;
            hijo.linterna = !hijo.linterna;

            for (uint32_t i = 0; i < arq.size(); i++) {
                // Checkear si esta de este lado
                if (nodo.arq[i] != nodo.linterna)
                    continue;

                hijo.arq[i] = !nodo.linterna;
                for (uint32_t j = i; j < arq.size(); j++) {
                    if (nodo.arq[j] != nodo.linterna)
                        continue;
                    hijo.arq[j] = !nodo.linterna;

                    // El movimiento siempre cuesta la menor de las velocidades
                    uint32_t costoExtra = max(arq[i], arq[j]);
                    uint32_t distancia = costo + costoExtra;

                    // Agregamos este hijo a la lista
                    agregarCandidato(dist, candidatos, hijo, distancia);

                    if (i != j)
                        hijo.arq[j] = nodo.linterna;
                }
                // Lo volvemos a como estaba
                hijo.arq[i] = nodo.linterna;
            }

            for (uint32_t i = 0; i < can.size(); i++) {
                // Checkear si esta de este lado
                if (nodo.can[i] != nodo.linterna)
                    continue;

                hijo.can[i] = !nodo.linterna;
                for (uint32_t j = i; j < can.size(); j++) {
                    if (nodo.can[j] != nodo.linterna)
                        continue;
                    hijo.can[j] = !nodo.linterna;

                    // El movimiento siempre cuesta la menor de las velocidades
                    uint32_t costoExtra = max(can[i], can[j]);
                    uint32_t distancia = costo + costoExtra;

                    // Agregamos este hijo a la lista
                    agregarCandidato(dist, candidatos, hijo, distancia);

                    if (i != j)
                        hijo.can[j] = nodo.linterna;
                }
                // Lo volvemos a como estaba
                hijo.can[i] = nodo.linterna;
            }

            for (uint32_t i = 0; i < arq.size(); i++) {
                // Checkear si esta de este lado
                if (nodo.arq[i] != nodo.linterna)
                    continue;

                hijo.arq[i] = !nodo.linterna;
                for (uint32_t j = 0; j < can.size(); j++) {
                    if (nodo.can[j] != nodo.linterna)
                        continue;
                    hijo.can[j] = !nodo.linterna;

                    // El movimiento siempre cuesta la menor de las velocidades
                    uint32_t costoExtra = max(arq[i], can[j]);
                    uint32_t distancia = costo + costoExtra;

                    // Agregamos este hijo a la lista
                    agregarCandidato(dist, candidatos, hijo, distancia);

                    hijo.can[j] = nodo.linterna;
                }
                // Lo volvemos a como estaba
                hijo.arq[i] = nodo.linterna;
            }
        }

        return res;
    }
\end{lstlisting}
\begin{lstlisting}[ basicstyle=\small]
    bool agregarCandidato(map<estado, uint32_t> &dist,
                          set<pair<uint32_t, estado>> &candidatos, estado &hijo,
                          uint32_t distancia) {
        auto s = dist.find(hijo);
        if (s != dist.end()) {
            // El hijo ya tiene una distancia
            // lo agregamos solo si la distancia resultante es menor
            uint32_t distVieja = s->second;

            if (distancia < distVieja) {
                // Hay que borrar la posicion en la cola de candidatos
                // para insertarlo de nuevo
                auto p = candidatos.find({distVieja, hijo});
                candidatos.erase(p);
                dist.erase(dist.find(hijo));
            } else {
                // La distancia anterior ya era mejor que venir por este camino
                return false;
            }
        }

        dist.insert({hijo, distancia});
        candidatos.insert({distancia, hijo});

        return true;
    }
\end{lstlisting}

\newpage
%-------------------------------------------------------------------
%-------------------------------------------------------------------
%-------------------------------------------------------------------
\subsection{Ejercicio 2}

\begin{lstlisting}[ basicstyle=\small]
      #include <math.h>
      #include <stdint.h>
      #include <iostream>
      #include <set>
      #include <vector>

      using namespace std;

      uint64_t peso;

      void balancear(uint64_t peso, vector<uint8_t> &izq, vector<uint8_t> &der) {
          // peso = sum(3^izq) - sum(3^der)

          bool carry = false;
          uint8_t exponente = 0;

          // Recorremos los digitos en base 3, empezando por el menos significativo
          while (peso > 0) {
              uint8_t digito = peso % 3;
              peso /= 3;

              // Le sumamos una pesa al digito anterior, y quedo un carry
              if (carry) {
                  digito += 1;
                  digito %= 3;

                  // Checkear si se genero otro carry mas
                  carry = digito == 0;
              }

              if (digito == 2) {
                  // A cada digito '2' del peso hay que sumarle 1

                  // Agregamos una pesa con este exponente
                  der.push_back(exponente);

                  // Actualizamos el peso
                  digito = 0;
                  carry = true;

              } else if (digito == 1) {
                  // Si quedo un 1 agregamos una pesa del otro lado, para
                  // contrarestarlo
                  izq.push_back(exponente);
              }

              exponente++;
          }

          if (carry)
              izq.push_back(exponente);
      }
\end{lstlisting}

\newpage
%-------------------------------------------------------------------
%-------------------------------------------------------------------
%-------------------------------------------------------------------
\subsection{Ejercicio 3}

\begin{lstlisting}[ basicstyle=\small]
#include <stdint.h>
#include <algorithm>
#include <iostream>
#include <map>
#include <set>
#include <vector>
#include <math.h>

using namespace std;

typedef struct mkp_object_t {
    uint64_t id;
    uint64_t weight;
    int64_t value;
} mkp_object;

vector<uint64_t> bins(3,0);
vector<mkp_object> objects;

uint64_t mkp(vector<uint64_t> bins, vector<mkp_object> const &objetos,
             vector<vector<uint64_t>> &resultIds) {
    // O(pi(bins) * count(objects) )

    // Si hay menos de 3 mochilas, consideramos las siguientes como
    // de tamano 0

    // Para cada combinacion de k_i pesos de las mochilas, con 0 <= k_i <=
    // capacidad(i)
    // calculamos el mayor valor conseguible ocupando exactamente ese espacio
    //
    // En cada combinacion, para cada 0 <= j <= cantidad de objetos,
    // calculamos el optimo usando los primeros j objetos.

    vector<vector<vector<vector<int64_t>>>> mejorValor(
        bins[0] + 1,
        vector<vector<vector<int64_t>>>(
            bins[1] + 1,
            vector<vector<int64_t>>(bins[2] + 1,
                                    vector<int64_t>(objetos.size() + 1, -1))));

    vector<vector<vector<vector<int64_t>>>> binObjeto(
        bins[0] + 1,
        vector<vector<vector<int64_t>>>(
            bins[1] + 1,
            vector<vector<int64_t>>(bins[2] + 1,
                                    vector<int64_t>(objetos.size() + 1, -1))));

    // 0 objetos en mochilas de tamano 0 tienen valor 0
    mejorValor[0][0][0][0] = 0;

    int64_t valorMax = 0;
    vector<uint64_t> sizesMax(3, 0);

    for (uint64_t w0 = 0; w0 <= bins[0]; w0++) {
        for (uint64_t w1 = 0; w1 <= bins[1]; w1++) {
            for (uint64_t w2 = 0; w2 <= bins[2]; w2++) {
                // Para cada primeros j objetos buscamos el máximo
                for (uint64_t j = 1; j <= objetos.size(); j++) {
                    const mkp_object &obj = objetos[j - 1];

                    // Decidimos si conviene poner el objeto en una mochila o no
                    // usarlo
                    int64_t valCandidato, valor = mejorValor[w0][w1][w2][j - 1];

                    valCandidato = obj.weight > w0 ? -1 :
                        mejorValor[w0 - obj.weight][w1][w2][j - 1] + obj.value;
                    if (valor < valCandidato) {
                        binObjeto[w0][w1][w2][j] = 0;
                        valor = valCandidato;
                    }

                    valCandidato = obj.weight > w1 ? -1 :
                        mejorValor[w0][w1 - obj.weight][w2][j - 1] + obj.value;
                    if (valor < valCandidato) {
                        binObjeto[w0][w1][w2][j] = 1;
                        valor = valCandidato;
                    }

                    valCandidato = obj.weight > w2 ? -1 :
                        mejorValor[w0][w1][w2 - obj.weight][j - 1] + obj.value;
                    if (valor < valCandidato) {
                        binObjeto[w0][w1][w2][j] = 2;
                        valor = valCandidato;
                    }

                    mejorValor[w0][w1][w2][j] = valor;
                }

                // Mantenemos el máximo valor visto
                if (valorMax < mejorValor[w0][w1][w2][objetos.size()]) {
                    valorMax = mejorValor[w0][w1][w2][objetos.size()];
                    sizesMax = {w0, w1, w2};
                }
            }
        }
    }

    // Escribimos los ids del resultado
    resultIds.resize(bins.size());

    for(uint64_t j=objetos.size(); j > 0; j--) {
        int64_t bin = binObjeto[sizesMax[0]][sizesMax[1]][sizesMax[2]][j];
        if (bin != -1) {
            resultIds[bin].push_back(objetos[j-1].id);
            sizesMax[bin] -= objetos[j-1].weight;
        }
    }

    // Como recorrimos los objetos desde el ultimo hay que invertir el array
    // (se pide ordenado)
    for (auto &v : resultIds) {
        reverse(v.begin(), v.end());
    }
    return valorMax;
}
\end{lstlisting}
