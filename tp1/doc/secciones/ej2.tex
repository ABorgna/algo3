\section{Problema 2: Problemas en el camino}

\subsection{Introducción}
	El siguiente obstáculo del grupo de arqueólogos consiste en reconstruir el estado de una balanza de dos platillos con una carga $P$ del lado izquierdo utilizando únicamente pesas de distintas potencias de 3. Además la solución propuesta debe tener complejidad temporal del orden $O(\sqrt(P))$
	\\ 

	Por la naturaleza del problema, donde nos interesa determinar únicamente la pertencia de cada potencia en el total de las que usamos para nivelar la balanza (y distinguir a qué lado de la balanza van) la representación del estado final de cada platillo por separado mediante conjuntos disjuntos resulta apropiada. Llamaremos $I$ al conjunto final de las potencias sobre el platillo izquierdo y $D$ a aquellas de la derecha. Además, como cada plato es contrapeso del otro, el estado de la balanza respecto de uno de los platillos se interpreta como su carga menos la del opuesto. 
	\\

	A partir de dichas representaciones, dado un valor de $P$, queremos encontrar dos conjuntos disjuntos de potencias de base 3 $I$ y $D$ tales que recreen el estado de la balanza original con una única carga $P$ a la a izquierda. En operatoria:
	\\

	$\sum D$ - $\sum I$ = $P$ \ \ $\wedge$ \ \ $I \cap D = \emptyset$ 
	\\ 
	
	Por ejemplo, para $P$ = 34 el par de conjuntos $\left \langle D,I  \right \rangle$ = $\left \langle \{27,9,1\},\{3\}  \right \rangle$ es solución porque \\
	$\{27,9,1\} \cap \{3\} = \emptyset \ \wedge \ 34 = \sum D - \sum I = (27+9+1)-(3)$.
	\\
	
	

\newpage 
\subsection{Solución y Correctitud}
	Para poder determinar $I$ y $D$ es necesario conocer la descomposición polinómica base 3 de $P$. El caso más simple de analizar es cuando $P$ es una potencia de 3 y entonces solamente requerimos de un único elemento para $D$ (es decir, el mismo $P$) y ninguno para $I$. En realidad este es un subcaso de otro más general: que la representación en base 3 de $P$ sea binaria (no haya dígitos que valen 2). Estos casos se pueden resolver fácilmente con $D$ como el conjunto de todos los factores de la descomposición polinómica. Formalizando, sea $P_{10}$ = $A_3$ = $\left \langle A_{ \left \lfloor{log(P)}\right \rfloor + 1 } \ ... \ A_1, A_0  \right \rangle$ por definición entonces \\
	
	$P$ = $\sum_{i = 0}^{\left \lfloor{log(P)}\right \rfloor + 1} A_i*3^{i}$ y nos alcanza con tomar $D$ = $\bigl ( \ \bigcup_{i=0}^{\left \lfloor{log(P)}\right \rfloor + 1} \{A_i\} \ \bigr ) \setminus \{0\} \ \ \wedge \ \ I = \emptyset $
	\\
	
	Sin embargo, hasta ahora obviamos qué sucede si hay coeficientes que valen 2 en la descomposición polinómica de $P$. Retornando a modo ilustrativo al contexto narrativo del problema, no poseemos múltiples pesas de un mismo peso. Es decir que, manteniendo por base la idea para el caso anterior, necesitamos alguna mecánica que garantice la existencia de una solución a costa de cargar valores en $I$ que complementen manipulaciones en $A_3$ para determinar $D$. 
	\\
	
	* cómo solucionar lo de los 2's (explicación breve de lo que hace el algoritmo con los carries) * 
	\\
	
\lstset{basicstyle=\large}
\begin{lstlisting}
    $\neg$hayCarry
    exponente $\leftarrow$ 0

    for digito in (P en base 3):
        if hayCarry
            digito $\leftarrow$ digito + 1
            digito $\leftarrow$ digito $mod$ 3
            hayCarry si digito es 0

        if digito == 2:
            izq.push_back(exponente);
            digito $\leftarrow$ 0;
            hayCarry

        else if digito == 1:
            der.push_back(exponente);
        
        exponente $\leftarrow$ exponente + 1
    
    if hayCarry:
        der.push_back(exponente)

\end{lstlisting}


\subsection{Complejidad}
	* explicación de cómo itera *

	* demo de que log(n) es O(sqrt(n)) *

\subsection{Análisis experimental}
