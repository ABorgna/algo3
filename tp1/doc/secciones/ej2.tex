\section{Problema 2: Problemas en el camino}

\subsection{Introducción}
	El siguiente obstáculo del grupo de arqueólogos consiste en reconstruir el estado de una balanza de dos platillos con una carga $P$ del lado izquierdo utilizando únicamente pesas de distintas potencias de 3. Además la solución propuesta debe tener complejidad temporal del orden $\mathcal{O}(\sqrt(P))$
	\\

	Por la naturaleza del problema, donde nos interesa determinar únicamente la pertencia de cada potencia en el total de las que usamos para nivelar la balanza (y distinguir a qué lado de la balanza van) la representación del estado final de cada platillo por separado mediante conjuntos disjuntos resulta apropiada. Llamaremos $I$ al conjunto final de las potencias sobre el platillo izquierdo y $D$ a aquellas de la derecha. Además, como cada plato es contrapeso del otro, el estado de la balanza respecto de uno de los platillos se interpreta como su carga menos la del opuesto.
	\\

	A partir de dichas representaciones, dado un valor de $P$, queremos encontrar dos conjuntos disjuntos de potencias de base 3 $I$ y $D$ tales que recreen el estado de la balanza original con una única carga $P$ a la a izquierda. En operatoria:
	\\

	$\sum I$ - $\sum D$ = $P$ \ \ $\wedge$ \ \ $I \cap D = \emptyset$
	\\

	Por ejemplo, para $P$ = 34 el par de conjuntos $\left \langle I,D  \right \rangle$ = $\left \langle \{27,9,1\},\{3\}  \right \rangle$ es solución porque \\
	$\{27,9,1\} \cap \{3\} = \emptyset \ \wedge \ 34 = \sum I - \sum D = (27+9+1)-(3)$.
	\\



\newpage
\subsection{Solución y Correctitud}
	Para poder determinar $I$ y $D$ es necesario conocer la descomposición polinómica base 3 de $P$. El caso más simple de analizar es cuando $P$ es una potencia de 3 y entonces solamente requerimos de un único elemento para $I$ (es decir, el mismo $P$) y ninguno para $D$. En realidad este es un subcaso de otro más general: que la representación en base 3 de $P$ sea binaria (no haya dígitos que valen 2). Estos casos se pueden resolver fácilmente con $I$ como el conjunto de todos los monomios de la descomposición polinómica, es decir, los dígitos de su representación ternaria. Formalizando, sea $P_{10}$ = $A_3$ = $\left \langle a_{ \left \lfloor{log(P)}\right \rfloor + 1 } \ ... \ a_1, a_0  \right \rangle$ por definición entonces \\

	$P$ = $\sum_{i = 0}^{\left \lfloor{log(P)}\right \rfloor + 1} a_i*3^{i}$ y nos alcanza con tomar $I$ = $\bigl ( \ \bigcup_{i=0}^{\left \lfloor{log(P)}\right \rfloor + 1} \{a_i*3^{i}\} \ \bigr ) \setminus \{0\} \ \ \wedge \ \ D = \emptyset $
	\\

	Sin embargo, hasta ahora obviamos qué sucede si hay coeficientes que valen 2 en la descomposición polinómica de $P$. Retornando a modo ilustrativo al contexto narrativo del problema, no poseemos múltiples pesas de un mismo peso. Es decir que, manteniendo por base la idea para el caso anterior, necesitamos alguna mecánica que garantice la existencia de una solución a costa de cargar valores en $D$ que complementen manipulaciones en $A_3$ para determinar $I$.
	\\

	La algoritmia de la solución trata en recorrer, desde el menos significativo, cada uno de los dígitos $a_i$ de $A_3$, la representación ternaria de $P$, y determinar para ambos conjuntos si $3^{i}$ pertenece a alguno. Como la base está implícita, únicamente guardamos los índices o exponentes $i$ a medida que los recorremos, quedando ordenados crecientemente, y nos encargamos luego de reproducir los verdaderos elementos de $I$ y $D$ elevando 3 a cada exponente. Entonces usamos secuencias $der$ e $izq$ que cumplen:
	\\

	$D = \bigcup_{j=0}^{|der|-1} \{3^{der_{j}}\}  $   \  \ \ \ (idem para $izq$ con $I$)
	\\

	Manteniendo la idea del primer caso, cuando el dígito $a_i$ sea 1 entonces agregamos el exponente $i$ al conjunto $I$, de ser 0 seguimos recorriendo. De este modo seguimos tomando a $A_3$ como base para $I$ aprovechando que tomar los términos de la descomposición polinómica de $P$ para $I$ y $D = \emptyset$ siempre cumple la condición $\sum I$ - $\sum D$ = $P$. Nos queda ver cómo deshacernos de los términos con coeficientes iguales a 2.

	Pero si $a_i$ es 2 lo que hacemos es contar ese exponente para $D$ y provocamos un acarreo hacia el próximo dígito $a_{i+1}$ que pasa a valer $(a_{i+1} + 1) \ mod \ 3$. Se puede ver que puede llegar a caer en otro dígito que ya valía 2, generando más acarreos hasta caer en un dígito que valga 1 (teniendo que generar otro acarreo por la idea del algoritmo) ó 0, 'tomando' el acarreo y sumando para $I$ la diferencia que creamos al agregar el exponente original $i$ en $D$, y poder seguir aplicando el mismo método en los próximos dígitos.

	Esto, visto desde la resta de ambos conjuntos, significa sumar $3^{i}$ tanto para $\sum I$ como $\sum D$. Por invariante de la iteración \footnote{Hasta ahora solamente nos ocupamos de los dígitos anteriores, por lo que $i \notin der \cup izq$}, si consideramos las sumas de ambos conjuntos como una descomposición polinómica base 3, este término valía 0 para $\sum D$ y $2*3^{i}$ para $\sum I$, quedando $3^{i}$ y $(2+1)*3^{i} = 3^{i+1}$ respectivamente y viendo además que de este modo no repetimos exponentes entre $I$ y $D$ ($3^i$ pertenece a $D$ pero no a $I$).
	\\
	
	Por lo tanto, distribuyendo restas algebráicamente válidas de la pinta ($3^{i}-3^{i}$) entre $I$ y $D$ pudimos mantener la idea que satisfacía originalmente $\sum I$ - $\sum D$ = $P$ partiendo de $D$ = $\emptyset$ e $I$ como los monomios de la descomposición polinómica base 3 de $P$, de tal modo que fuimos 'pateando' los términos con coeficientes de valor 2 a términos de mayor grado y manteniendo los dos conjuntos con intersección vacía. En el pseudocódigo siguiente se ve más claro incluso por qué nunca se cuenta el mismo exponente para ambos conjuntos.
	
	\newpage

\lstset{basicstyle=\large}
\begin{lstlisting}
    hayCarry $\leftarrow$ false
    exponente $\leftarrow$ 0

    for digito in (P en base 3):
        if hayCarry:
            digito $\leftarrow$ digito + 1
            digito $\leftarrow$ digito $mod$ 3
            hayCarry $\leftarrow$ true if digito es 0

        if digito es 2:
            agregar exponente al final de der
            digito $\leftarrow$ 0
            hayCarry

        else if digito es 1:
            agregar exponente al final de izq

        exponente $\leftarrow$ exponente + 1

    if hayCarry:
         agregar exponente al final de izq

\end{lstlisting}

	%Proposición:
	%\\
	%Q($j$): Para todo dígito $a_j$ iterado, con $0 \leq j < \left \lfloor{log(P)}\right \rfloor + 1$, tras su iteración las secuencias $der$ e $izq$ son disjuntas y se cumple que:
	%\\
	%
	%$\sum_{i=0}^{|der|-1} 3^{der_i} + \beta(hayCarry)*3^{j+1} - \sum_{i=0}^{|izq|-1} 3^{izq_i} = \sum_{i=0}^{j} a_i*3^{i} $
	%\\
	%
	%Probémoslo por inducción en $j$. Sea el caso base:
	%
	%Q($0$): En la primer iteración se comienza sin $carry$ siempre.
	%
	%\begin{itemize}
	%\item Si $a_0$ = 0 entonces $der$ = $izq$ = $[]$ y no se genera $carry$. Por lo tanto vale la igualdad porque ambos lados son nulos y ambas secuencias siguen vacías y por lo tanto disjuntas.
	%\item Si $a_0$ = 1 entonces $izq$ = $[0]$, $der$ = $[]$ y tampoco se genera $carry$. $der$ e $izq$ siendo disjuntas y vale que $3^{0} + \beta(false)*3^{1} - 0 = 1 = 1*3^{0}$-
	%\item Si $a_0$ = 2 entonces $der$ = $[0]$, $izq$ = $[]$ y se genera $carry$. Al igual que en el caso anterior, siguen siendo disjuntos. También vale la igualdad porque $0 + \beta(true)*3^{1} - 3^{0} = 3 - 1 = 2*3^{0} = 2$
	%\end{itemize}
	%
	%Por lo tanto, en todos los casos sabemos que se preservan ambas condiciones.
	%\\
	%
	%Paso inductivo:
	%Asumo válido Q($n-1$), quiero ver que vale Q($n$).
	%En la iteración anterior puede haberse, o no, producido $carry$. Separamos en casos, empezando con el caso en que no hay:
	%
	%\begin{itemize}
	%\item Si $a_n$ = 0 entonces no se modifica ninguna de las secuencias ni se genera $carry$, por HI entonces estas condiciones cumplen la igualdad y la disjunción.
	%\item Si $a_n$ = 1 entonces $der = pre(der)++[n] $, $izq = pre(izq)$ y no se genera $carry$. Por HI sabemos que vale:   \\
	%$\sum_{i=0}^{|der|-2} 3^{der_i} - \sum_{i=0}^{|izq|-1} 3^{izq_i} = \sum_{i=0}^{n-1} a_i*3^{i} $
	%\\
	%Usando que $3^{|der|-1} = 3^{n} = a_{n}*3^{n}$, sumando de ambos lados:
	%\\
	%$\sum_{i=0}^{|der|-2} 3^{der_i} + 3^{|der|-1} - \sum_{i=0}^{|izq|-1} 3^{izq_i} = \sum_{i=0}^{n} a_i*3^{i} + 3^{n} $
	%\\
	%$\sum_{i=0}^{|der|-1} 3^{der_i} - \sum_{i=0}^{|izq|-1} 3^{izq_i} = \sum_{i=0}^{n+1} a_i*3^{i}$
	%Que es
	%\item Si $a_j$ = 2 entonces
	%\end{itemize}

\subsection{Complejidad}
	* explicación de cómo itera *

	* demo de que log(n) es O(sqrt(n)) *

\subsection{Análisis experimental}
